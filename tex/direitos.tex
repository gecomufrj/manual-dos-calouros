\ssection{Direitos e Deveres do Estudante}
	Esta é aquela seção que nós adoraríamos ter lido no início do curso. Nem todo aluno conhece, mas existem algumas resoluções que protegem nossos direitos.
    Uma resolução (``do CEG'', ``da Poli''...) é um documento que determina os procedimentos para lidar com alguma determinada questão.
    Através delas, podemos ``escalar as instâncias'' -- professor, coordenação do curso/departamento, diretoria da Poli -- para garantir nossos direitos.
    
    Duas das mais controversas são a Resolução CEG 4/96 (leia-se ``Resolução Nº 4 do Conselho de Ensino de Graduação, no ano de 1996'') e a resolução 1/17 da Poli. A primeira de prazos para requerimento e concessão de vista de prova e revisão de nota e a segunda, de prazos para entrega das notas parciais antes da prova final.
    Alguns professores não seguem estes prazos, então é muito importante manter a calma e a compostura na hora de conversar com eles sobre isso, pois é um direito seu, mas nem sempre é respeitado. A última coisa que você quer é criar um clima estranho, afinal, seus professores são responsáveis por suas provas, notas e aprovações; sabemos que muitas vezes eles erram, mas é de extrema importância que saibamos abordar esse assunto de forma respeitosa e correta.
    
    As resoluções da Poli podem ser encontradas no site da Poli \underline{\href{http://www.poli.ufrj.br/resolucoes.php}{Poli $>$ Graduação $>$ Resoluções}}.
    As do CEG podem ser encontradas em: \underline{\href{https://xn--graduao-2wa9a.ufrj.br/index.php/469-ceg-resolues}{PR1 $>$ CEG $>$ Resoluções}}. É interessante notar que nas resoluções do CEG, elas podem ser classificadas por data ou por tema.
    Também é bom se atentar ao fato de que uma resolução mais recente pode invalidar uma anterior.
