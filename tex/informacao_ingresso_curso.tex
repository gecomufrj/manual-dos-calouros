\ssection{Processos burocráticos no curso}

    \ssubsection{Dispensa de disciplinas}
        Esse procedimento é usado por alunos que já cursaram disciplinas equivalentes às do nosso curso em outras universidades, em intercâmbio ou em outros cursos de graduação da própria UFRJ e desejam ser dispensados de cursar essas mesmas disciplinas aqui no curso de Engenharia Eletrônica e de Computação da UFRJ. Em geral, duas exigências são feitas: 
            \begin{itemize}
                \item Carga horária de pelo menos 75\% da disciplina análoga na UFRJ
                \item Ementa da matéria equivalente à análoga na UFRJ
            \end{itemize}
            
        Tais requisitos são avaliados pelo coordenador do curso. Dessa forma, é necessário apresentar ao coordenador a ementa da disciplina e o histórico do aluno na instituição, onde conste a aprovação do mesmo na matéria a pedir a dispensa. Além disso, há um formulário de dispensa que será preenchido em conjunto com o coordenador, mas que você deve anexar aos documentos citados anteriormente. Esse formulário pode ser encontrado em \href{http://www.poli.ufrj.br/arquivos/secretaria/Novo\_formulario\_de\_Dispensa\_de\_Disciplinas\_Protocolo\_da\_POLI.pdf}{Site da Poli $>$ Secretaria Online $>$ Formulário de Abertura de Processo de Dispensa de Disciplinas}.
        
        Vale ressaltar que em tal processo, as notas das disciplinas a serem dispensadas não são registrados no Histórico Escolar do aluno e nem são contabilizadas no cálculo do Coeficiente de Rendimento (CR) do aluno.
    
    \ssubsection{Trancamento de disciplina}
        O trancamento de disciplina é feito em casos em que o aluno tem uma inscrição efetivada na matéria, mas deseja ``cancelar'' a mesma. Em geral, o período de trancamento se estende por até um mês após o início do período, de forma que os alunos possam ir nas primeiras aulas, ver como é a ementa, a forma com que o professor ministra a disciplina e etc. Pode ser feito de duas maneiras:
            
            \begin{itemize}
                \item Pela Secretaria Acadêmica do Bloco H (em casos de alunos que estejam no primeiro período, que não se inscreveram pelo Siga)
                \item Pelo Portal do Aluno (o famoso Siga UFRJ): \href{https://portal.ufrj.br}{https://portal.ufrj.br}
            \end{itemize}
    
    \ssubsection{Trancamento de matrícula}
        O trancamento de matrícula é feito em casos em que o aluno necessita interromper suas atividades acadêmicas por tempo determinado e este é definido como \textit{trancamento solicitado}.Todas as informações a respeito de trancamento de matrícula podem ser encontradas na \href{http://caxias.ufrj.br/images/Resolucao_CEG_03_08.pdf}{Resolução CEG 03/2008}.
        
        Só é possível dar entrada em tal processo caso o aluno já tenha cursado, com aproveitamento, no mínimo 12 créditos. Além disso, a matrícula pode ficar trancada até quatro períodos e, caso o aluno não solicitar o destrancamento até o período máximo, a matrícula na UFRJ é cancelada por abandono. Há também outros detalhes nas regras, como estender a quantidade máxima de períodos, por conta de impedimentos físicos e etc; estas informações mais detalhadas podem ser lidas na resolução citada acima.
        
        O prazo para tal processo é sempre maior do que o prazo para trancamento de matérias, podendo se estender até o meio do período. Para dar entrada no processo de trancamento, basta ir até a secretaria acadêmica do curso, localizada no bloco H.
        
        Por fim, como informação adicional e resumida da resolução, há também o \textit{trancamento automático}, que ocorre em casos específicos, que são:
        
        \begin{itemize}
            \item O aluno com matrícula ativa que não efetuar a inscrição de disciplinas no prazo determinado pelo Calendário Acadêmico do Curso (inscrição e alteração de disciplinas)
            \item O aluno que, ao fim do período letivo, apresentar um coeficiente de rendimento (CR) igual a zero.  
        \end{itemize}
        
    \ssubsection{Registro de atividades complementares especiais (ACEs)}
        As ACEs se traduzem em horas para o seu Boletim de Orientação Acadêmica (BOA) e são necessárias para que você se forme. Nossa grade curricular prevê o cumprimento de 405 horas extracurriculares, que podem ser horas gastas em atividades como administração de empresa júnior, iniciação científica, monitoria, participação em eventos, voluntariado, dentre outras. 
        
        Para tanto, após o cumprimento de tais horas, você tem até dois anos para solicitar o registro das mesmas no seu BOA, ou perderão a validade. O processo para contabilizar essas horas é bem simples: basta você preencher um formulário da Escola Politécnica referente à atividade que você deseja registrar e anexar algum documento que comprove o cumprimento da mesma (um exemplo pode ser a participação em eventos na UFRJ que, em geral, emitem certificados nominais). Feito isso, é necessário entregar toda a documentação na secretaria do curso para que o coordenador possa verificar as informações e assinar. Assim que o documento é assinado, ele é encaminhado para o setor da Escola Politécnica que cadastra as horas complementares no sistema e, no término do processo, você recebe um e-mail, com assunto ``Cadastramento de RCS'', informando a situação do mesmo.
        
         Todos os formulários de registro de ACES podem ser encontrados em \href{http://www.poli.ufrj.br/graduacao\_aces.php}{Site da Poli $>$ Graduação $>$ Atividades Complementares Especiais}.
    
    \ssubsection{Estágio}
        Para conclusão do curso, o aluno deve cumprir 160 horas de estágio obrigatório, que pode ser feito em ambiente interno da UFRJ ou em empresas e instituições que mantenham convênio com a universidade - há uma lista com os convênios com a UFRJ, que pode ser encontrada em \href{http://graduacao.ufrj.br/images/stories/_pr1/pdf/estagio/Convenios-para-Estagios.pdf}{Site da Pró-Reitoria de Graduação (PR-1) $>$ Convênios para Estágios}. 
        
        Além disso, a Escola Politécnica tem uma resolução que regulamenta o ingresso do aluno em um estágio, seja ele não-obrigatório ou obrigatório. Essa seção está dividida em itens, para permitir um melhor entendimento do assunto, que é causa de muitas dúvidas. Para tanto, vamos resumir o que é preciso saber, mas caso você queira saber mais, pode consultar a resolução aqui: \href{http://www.poli.ufrj.br/arquivos/resolucoes/n03de290513_Altera_Resolucao02de2009_Normas_para_o_Estagio_de_Estudantes_na_Escola_Politecnica.pdf}{Site da Escola Politécnica $>$ Resoluções $>$ Normas para o Estágio}.
    
        \ssubsubsection{Não-obrigatório}
            Antes de qualquer coisa, vale dizer que essa modalidade de estágio \textbf{não é válida para a matéria obrigatória de estágio presente no Projeto Pedagógico do Curso, que contabiliza créditos} e é opcional, contabilizando apenas horas de Atividades Complementares Especiais (ACEs). 
        
            Em geral, o aluno se vê nessa situação quando recebe uma proposta de estágio, mas não tem o pré-requisito necessário para cumprir um estágio obrigatório, que é ter concluído, pelo menos, 70\% do curso. Porém, também há pré-requisitos para fazer um estágio não-obrigatório, que são:
                \begin{itemize}
                    \item 50\% da quantidade total de créditos do curso, incluindo matérias obrigatórias, condicionadas, restritas e livres.
                    \item Aprovação no ciclo básico do curso (referente aos quatro primeiros períodos)
                \end{itemize}
        
        
        \ssubsubsection{Obrigatório}
            Para fazer essa modalidade de estágio, o aluno tem que ter cumprido, pelo menos, 70\% do curso.
        
        \ssubsubsection{Plano de estágio}
            Para começar o processo burocrático de reconhecimento de um estágio pela UFRJ, o primeiro documento que deve ser preenchido é o plano de estágio, que consiste em um breve resumo das atividades que o aluno irá exercer, a empresa em que irá fazer o estágio, ou o projeto, no caso de estágio interno e o supervisor responsável pelo discente. 
            
            É obrigatório que o supervisor do estágio seja um profissional da área do curso, ou seja, um engenheiro ou profissional da área da Computação. Para comprovar isso, o aluno deve informar no plano de estágio o número do CREA ou apresentar uma cópia do diploma do supervisor. No caso específico do supervisor de estágio ter se formado na UFRJ, o membro da Comissão de Estágio consegue verificar a formação do mesmo na área através do SIGA.
            
            Esse plano de estágio será avaliado por um membro da Comissão de Estágio do Curso, atualmente professores Fernando Baruqui e Joarez Bastos, e, caso esteja tudo correto, será assinado e devolvido a você. Após obter essa assinatura, será necessário entregá-lo na Diretoria Adjunta de Ensino e Cultura (DAEC), junto ao Termo de Compromisso de Estágio (TCE).
        
            Os formulários podem ser encontrados no site da Escola Politécnica, em:
            
            \begin{itemize}
                \item \href{http://www.poli.ufrj.br/arquivos/resolucoes/Anexo_Resolucao_n_03_de_29_05_2013_Plano_de_Estagio_Obrigatorio_ambiente_interno.pdf}{Resoluções $>$ Anexo: Plano de Estágio para Ambiente Interno}
            
                \item \href{http://www.poli.ufrj.br/arquivos/resolucoes/AnexoResolucao_n03de29052013_Plano_de_Estagio_empresa.pdf}{Resoluções $>$ Anexo: Plano de Estágio para Empresas ou Instituições Conveniadas}
            \end{itemize}
        
        \ssubsubsection{Termo de compromisso de estágio (TCE)}
            O TCE é o instrumento jurídico que irá regulamentar a relação entre a concedente (em geral, a empresa), o aluno e a Escola Politécnica. Esse documento deve ser assinado por todas as partes, sendo a UFRJ a última a assinar, pois há uma validação por parte da universidade do termo. 
        
            Para estágios em ambiente interno, há um formulário padrão, que pode ser encontrado em \href{http://www.poli.ufrj.br/arquivos/resolucoes/Anexo_Resolucao_n03de29_05_2013_Modelo_de_TCE_ambiente_interno.pdf}{Resoluções $>$ Anexo: Termo de Compromisso de Estágio em Ambiente Interno}.
        
            Já para o caso de empresas conveniadas, há um modelo padrão, que pode ser usado como base, mas não é obrigatório, que pode ser baixado pelo seguinte \textit{link} no site da Poli:  \href{http://www.poli.ufrj.br/arquivos/estagio/TCE_Externo.doc}{Graduação $>$ Bolsas e Estágios $>$ Estágio $>$ Formulários e modelos de documentos $>$ Modelo de TCE externo}.
        
        \ssubsubsection{Relatório de atividades}
            Segundo a \href{https://xn--graduao-2wa9a.ufrj.br/images/stories/_pr1/pdf/estagio/lei_do_estagio_11788de2008.pdf}{Lei dos Estágios}, é necessário a elaboração de um relatório de atividades a cada seis meses. Tal documento é elaborado pelo supervisor do discente, similar a uma avaliação das atividades do estagiário e é utilizado para solicitar o registro das horas de ACEs, em caso de estágio não-obrigatório, ou para contabilizar os créditos da matéria de estágio obrigatório do fim do curso.
        
            Os formulários podem ser encontrados no site da Escola Politécnica, em
            
            \begin{itemize}
                \item \href{http://www.poli.ufrj.br/arquivos/resolucoes/Anexo_Resolucao_n03de29052013Relatorio_de_Avaliacao_de_Estagio_ambiente_interno.pdf}{Resoluções $>$ Anexo: Relatório de Avaliação de Estágio em Ambiente Interno}
            
                \item \href{http://www.poli.ufrj.br/arquivos/resolucoes/AnexoResolucao_n03de29052013Relatorio_de_Avaliacao_de\%20Estagio_empresa.pdf}{Resoluções $>$ Anexo: Relatório de Avaliação de Estágio em Ambiente Externo}
            \end{itemize}
            
            \ssubsubsection{Como entrar com um pedido de estágio?}
            
            Em primeiro lugar, é necessário verificar se, em caso de ambiente externo, a empresa possui vínculo com a universidade, através do \textit{link} disponibilizado no início da seção ``Estágio''. Caso a empresa não possua esse convênio, será necessário fazê-lo. As instruções e o modelo do documento para celebração do convênio pode ser encontrada no \href{https://xn--graduao-2wa9a.ufrj.br/images/stories/_pr1/pdf/estagio/Minuta---Convenio-Empresa-Privada-2017-_Prof-Eduardo_.pdf}{Site da PR-1 $>$ Estágios $>$ Convênio com Empresas Privadas}.
            
            Para o ingresso do aluno em um estágio, há uma série de procedimentos a serem cumpridos. São eles:
            
            \begin{itemize}
                \item Preenchimento do plano de estágio ou supervisor na empresa ou pelo coordenador do projeto (em caso de estágio interno)
                \item Validação do plano por um integrante da Comissão de Estágio do curso ou pelo coordenador
                \item Assinatura do TCE, por todas as partes (é um formulário próprio da Poli, em caso de estágio interno)
                \item Entrega do plano de estágio e do TCE na Diretoria Adjunta de Ensino e Cultura (DAEC)
                \item Aguardar o responsável da DAEC entrar em contato para retirada do TCE
            \end{itemize}
        
        
        
        