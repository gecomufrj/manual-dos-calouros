\ssection{Alimentação}
	\ssubsection{Bandejão}
    O R.U. (Restaurante Universitário), também conhecido popularmente como \textit{bandejão}, é, provavelmente, onde você irá almoçar nos dias em que estiver no Fundão (ou onde você, provavelmente, irá jantar se residir no campus [alojamento, vila...]).
    
    Com um preço atrativo de apenas R\$2,00, o Fundão conta com 3 unidades de R.U.:
	\begin{itemize}
    	\item No prédio da Faculdade de Letras
        \item No próprio CT (Centro de Tecnologia)
        \item Central (Prédio unicamente utilizado para serviços do bandejão)
    \end{itemize}
    
    O bandejão do nosso prédio (CT), que fica localizado no 1\textsuperscript{\underline{o}} andar do bloco A (ver mapa do CT em ``O Centro de Tecnologia"), é o menor do fundão. Por isso, o número de vagas para as refeições é limitado por meio de um sistema de agendamentos. Às 9h30 da manhã (para o almoço) e às 16h30 (para o jantar) o servidor do R.U. libera o agendamento. Devido ao número limitado de vagas, o agendamento se esgota 1 ou 2 minutos após o seu tempo de abertura (para o almoço). Portanto, se você realmente preferir almoçar no R.U. do CT, é recomendado realizar o agendamento pelo site ao invés do aplicativo (calma, ainda vamos falar dele). Além disso, recomendamos ter seu CPF salvo no \textit{clipboard} (famoso Ctrl-C) do seu celular e botar um alarme para às 9h28 para realmente não perder o horário.
    
    O bandejão da Faculdade de Letras é o R.U. mais perto do bloco onde estudamos no CT (bloco H). Por ser capaz de atender a uma demanda maior de alunos, o sistema de agendamentos não foi implementado nesse bandejão, o que causa longas filas em sua entrada; o tempo de fila varia muito. Como o horário de funcionamento do bandejão para almoço começa às 11h00, os estudantes já começam a formar a fila por volta das 10h40. Chegando na fila no intervalo compreendido entre [10h40, 11h50], o tempo de espera costuma variar entre 20-30 minutos. Chegando na fila no intervalo compreendido entre [11h50, 13h00], o tempo de espera costuma ser de 60 minutos. Tome muito cuidado para não acabar perdendo a sua aula das 13h00 por conta do bandejão!
    
    O Bandejão Central é o maior de todos. É lá que toda a comida dos R.U.s é produzida (inclusive a do bandejão do campus Praia Vermelha). O prédio do Central fica localizado do outro lado do Fundão, perto do Hospital Universitário (H.U.) e da Escola de Educação Física e Dança (E.E.F.D), ou seja, caso queira fazer a sua refeição lá, terá que pegar um ônibus circular do Fundão - qualquer um que esteja indo no sentido "Estação UFRJ" (para mais detalhes, ver seção "Ônibus"). Por ser longe de onde estudamos, o tempo de ida e volta acrescido do tempo de espera na fila e do almoço geralmente extrapola a uma hora total que normalmente temos para almoçar. Portanto, avalie bem o seu tempo disponível antes de embarcar no circular. No entanto, mesmo que, no final das contas, o processo de almoçar no R.U. Central demore mais do que nos outros R.U.s, vale dizer que, por ter uma capacidade muito maior e por ter 2 filas, o tempo de espera é significativamente menor do que o do R.U. letras.
    
    Para saber qual é o menu do dia basta baixar o \textit{app} do bandejão: Cardápio RU-UFRJ. Lá, é possível ver os menus de almoço e jantar de todos os dias da semana. O \textit{app} atualiza os cardápios toda segunda, pela manhã.
    
    \subsubsection {Dicas}
    Procure levar o dinheiro trocado. \textbf{EVITE AO MÁXIMO} pagar o bandejão com uma nota de R\$50,00
    No caso de só ter R\$50,00, \textbf{implore} para quem estiver no caixa não te encher de moedas.
    
    Por último, mas não menos importante: \textbf{NÃO FURE FILAS}
    
    \newpage
    \ssubsection{\textit{Trailers}}
    Ao longo do 1\textsuperscript{\underline{o}} andar do CT, temos vários \textit{trailers} (geralmente um por bloco) vendendo diversos tipos de alimentos, desde doces a refeições. As refeições são ideais para quando você não tem tempo suficiente para ficar na fila do bandejão.
    
    \begin{itemize}
    \item Árabe
    
    Considerado por muitos um dos melhores \textit{trailers} do CT, apesar de ter um preço mais elevado em relação aos outros, é um bom lugar para comer com mais calma. Todo dia há dois cardápios, o trivial e o árabe. O prato árabe sempre vem com alguma pasta para comer com o pão árabe que vem de acompanhamento.
    
    \item Baptista
	
    Localizado no bloco B do CT, o Baptista conta com uma variedade de alimentos. É um bom lugar para tomar um açaí de maior qualidade do que geralmente se encontra no CT (mais caro) e dividir uma porção de batatas fritas com os amigos.

    \item Bom Gosto Lanches
    
    Também popularmente conhecido como ``PF do F'', justamente por ser um dos melhores PFs do CT. Lá, você paga um preço fixo pela proteína principal (Frango, Carne, Omelete, Estrogonofe) e tem direito a adicionar quantos itens da cartela quiser, tais como: arroz, feijão, batata frita, purê de batata, salada e farofa. Além disso, caso escolha o omelete, é possível personalizá-lo com queijo, presunto, tomate, salsinha, entre outros.
    
    \item Paulinho dos Doces
	
    Paulinho dos doces é um trailer localizado entre os blocos D e C. É conhecido por ter uma variedade muito grande de doces, mas também por ser caro. \textbf{DICA:} Bom lugar para trocar notas grandes.

    \item Yakisoba/China
    
    Tem um pão de queijo para aquele lanche e um yakisoba rápido para quando tiver com pressa para almoçar.
    
    
    \end{itemize}
    
    \ssubsection{Restaurantes}
    
    \begin{itemize}
    \item Burguesão
    
    Talvez seja o restaurante mais próximo do nosso bloco. É considerado caro por muitos alunos, mas é bom para comprar um café ou um pão de queijo sem ter que se deslocar para o bloco C, apesar de ainda ser mais caro do que no Yakisoba, como descrito acima.
    
     \item Grêmio da Coppe
    
    O Grêmio é o melhor lugar para você ir quando quer almoçar com os amigos, para dividir um prato. Em geral, é bom estar com tempo para ir lá, pois costuma ser bem cheio no horário do almoço. A maioria dos pratos serve duas ou mais pessoas.
    Sua localização é um pouco distante do CT, pois é atrás do bloco H, próximo à entrada do CT2.
    
    \item Projectus
    
    O Projectus é um restaurante \textit{self-service} atrás do bloco F. A comida é muito boa, sempre tem muita variedade e também opções vegetarianas. Porém, o preço do quilo é bem elevado, o que faz com que não seja a primeira opção de muitos alunos.
    
    \item Kilowatts
    
    É um \textit{self-service} muito semelhante ao Projectus e fica atrás do bloco H, em direção ao bloco I. Também é um restaurante com um preço mais elevado, mas a comida tem uma excelente qualidade.
    
    \item Restaurante do CT2
    
    É um \textit{self-service}, assim como os dois citados acima, mas com um preço mais em conta, o que leva muitos alunos a almoçarem por lá.
    \end{itemize}
    
    \ssubsection{Quiosques do segundo andar}
    
    \begin{itemize}
        \item Dilcinha
        
            Localizado no bloco C, tem os mais variados tipos de comida. Há quentinha, salgados, refrescos e uma quantidade relevante de doces e bolos, além de um café barato.
        
        
        \item Dona Graça e Seu Augusto
        
            Localizado entre os blocos G e H, é o que temos de mais próximo para comermos aquele lanche rápido, apesar de não ser muito barato. Há opções veganas e integrais de salgados, além de um hambúrguer que é famoso entre os alunos dos dois blocos (é possível acrescentar molho e batata palha no hambúrguer). 
            
            Além disso, há uma máquina de café, com várias opções diferentes, como café com chocolate; são deliciosos, mas caros. É um bom pedido para um dia de chuva, pois, se você puder sentar por lá, ainda vem com uma água com gás e um biscoito amanteigado.
    \end{itemize}
    
    