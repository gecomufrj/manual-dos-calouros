\ssection{Atividades extracurriculares}
Talvez vocês não saibam, mas não é só de matérias obrigatórias que nós, alunos, vivemos no Fundão. Nossa grande área, a Engenharia, prevê a aplicação, na prática, do conhecimento adquirido em sala. É claro que temos aulas práticas, principalmente de Eletrônica, como foi visto na seção sobre o nosso curso. Porém, há diversas atividades legais, inclusive desenvolvidas por alunos, que você pode fazer parte. Vale ressaltar também que as atividades exercidas podem ser contabilizadas como horas extracurriculares, as Atividades Complementares Especiais (ACEs).

Nessa seção, iremos apresentar algumas das principais atividades que você pode exercer dentro do nosso curso. Lembramos que, acima de tudo, é importante fazer aquilo que gostamos e, exatamente pela grande diversidade de projetos dentro do CT, não entraremos em detalhes sobre cada um deles. Estaremos à disposição para podermos conversar e também tirar dúvidas.

Lembramos também que as iniciativas que serão apresentadas são só alguns exemplos, mas há diversas outras no Fundão. O que não falta é algo para que você se identifique e fique com vontade de participar!

\subsection{Atlética de Engenharia}

A atlética se chama AAAEP (Associação Atlética Acadêmica Escola Politécnica). Os atletas são de todos os cursos de engenharia do Fundão. Um fato importante é que somos invictos até atualmente (2018), desde que foi criada a instituição (2011), ganhamos todos os troféus gerais do Intereng. 

Se você tem interesse por esportes, há seletivas para diversos esportes ao longo do período, de forma que você pode fazer parte da equipe de algum deles, como por exemplo, futsal, futebol de campo, basquete, vôlei e etc. Geralmente, as datas são divulgadas no Instagram (\href{https://www.instagram.com/atleticaengenhariaufrj/}{@atleticaengenhariaufrj}) e no Facebook (\href{https://www.facebook.com/atleticaengenhariaufrj/}{fb.com/atleticaengenhariaufrj}). Além disso, temos nossa própria torcida que nos alenta nos jogos universitários. Assim, se você tem interesse por esportes em si, mas não pensa em praticá-los, a Cachorrada do Fundão também tem seletivas, que geralmente são divulgadas no Facebook (\href{https://www.facebook.com/CachorradaDoFundao/}{fb.com/CachorradaDoFundao}). 

A sede da atlética fica localizada na entrada do bloco D, lá são vendidos vários itens como canecas, camisas, copos etc.

\ssubsection{Empresas Júnior}
Geralmente são equipes/empresas formadas por alunos e ex-alunos que executam projetos externos, sejam esses projetos contratados por empresas ou projetos voluntários criados pelos alunos. Há alguns exemplos de grande destaque, como a Fluxo Consultoria, uma empresa júnior que executa projetos de consultoria para empresas externas à UFRJ. Além da Fluxo, existe a Enactus UFRJ, que é uma organização sem fins lucrativos que visa o empoderamento das pessoas através de projetos sociais feitos no Rio de Janeiro, principalmente em comunidades, como o Caju.

\ssubsection{Equipes de Competição}
São iniciativas geralmente criadas por alunos onde são desenvolvidos, na prática, projetos que visam participação em alguma competição. Há mais projetos do que podemos pensar em listar aqui para vocês, mas, por exemplo, temos certeza de que alguns de vocês já ouviram falar das competições de robóticas que existem no Brasil e no mundo; ou talvez as competições de lançamento de foguetes, que geralmente acontecem nos Estados Unidos. Enfim, o que não falta é oportunidade e variedade para você escolher o que tem vontade de fazer.

Em geral, todas as equipes de competição fazem processos seletivos, que podem acontecer em diferentes intervalos de tempo, donde o mais comum é ter um processo por período, ou seja, dois por ano.

A parte interessante dessas equipes é que você consegue entrar em contato, mesmo nos primeiros períodos, com problemas práticos de engenharia, o que inclui adquirir conhecimentos em várias áreas, dependendo da equipe. Abaixo, listaremos algumas das principais equipes que temos no Fundão, junto com uma breve descrição.

\begin{itemize}
\item Equipe Ícarus UFRJ: carro movido a combustível (gasolina)
\item Equipe Solar Brasil: barco movido à energia solar
\item Minerva Aerodesign: aeromodelismo
\item Minerva Baja: carro \textit{off-road}
\item Minerva Bots: robôs de diferentes categorias (seguidor de linha, combate de diferentes pesos, sumô)
\item Minerva E-Racing: carro elétrico
\item Minerva Náutica: barcos controlados remotamente ou autônomos
\item Minerva Rockets: foguetes e sistemas similares a satélites
\item UFRJ Nautilus: submarino autônomo
\end{itemize}

\ssubsection{Iniciação Científica}
Na Iniciação Científica, você tem a oportunidade de ter o primeiro contato com o meio acadêmico, com a pesquisa em laboratório. Todos os laboratórios citados anteriormente possuem projetos que aceitam alunos da graduação, muitas vezes sem exigir experiência prévia. Durante o projeto, você é orientado por um professor e o intuito é aprender, tendo contato com conceitos que vão além dos abordados em sala de aula e que podem até levar à publicação de um artigo sobre o trabalho que você desenvolveu.

Essa atividade pode ser remunerada caso o laboratório possua disponibilidade de bolsa, no valor de R\$400,00 por mês. Se não tiver bolsa no momento, vale uma conversa com o professor caso o assunto do trabalho seja algo que você goste, porque pode ser que você consiga a bolsa depois.

Se você tiver bolsa de Iniciação Científica, você precisa apresentar seu trabalho na SIAc (Semana de Integração Acadêmica), que acontece todo ano em meados de outubro. Nela, são apresentados todos os trabalhos que estão sendo desenvolvidos na universidade e você pode assistir às apresentações, mesmo não fazendo Iniciação Científica. 
\ssubsection{Monitoria}
Depois de passar por uma matéria que você gostou, você pode procurar o professor para saber se existe vaga de monitoria. Essa atividade pode ser com bolsa ou voluntária, depende de disponibilidade. O trabalho do monitor varia muito dependendo do professor da matéria, mas, geralmente, envolve tirar dúvidas dos alunos que estão cursando a matéria e corrigir listas de exercícios. 