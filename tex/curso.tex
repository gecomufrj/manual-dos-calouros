\ssection{Nosso Curso}
	\ssubsection{Estrutura do Curso}
    	A Engenharia Eletrônica e de Computação é um curso que apresenta algumas especializações diferentes: Controle, Computação, Eletrônica, Sistemas Digitais e Telecomunicações. Essas áreas apresentam correlação entre si e não temos especialização explícita no diploma (e.g. Telecomunicações exige um conhecimento ferramental de Computação). 
    	
    	Você pode encontrar a grade curricular do nosso curso no \href{https://siga.ufrj.br/sira/temas/zire/frameConsultas.jsp?mainPage=/repositorio-curriculo/ACE1A6AC-92A4-F79D-3494-03B00781F450.html}{Sistema Integrado de Gestão Acadêmica (SIGA)} - \textit{link} que você pode encontrar pelo \textit{Google}, enquanto não tem o \textit{login}. Além disso, preparamos um breve resumo sobre cada habilitação e suas áreas de atuação.
        
        \ssubsubsection{Controle}
        	A área de Controle é a habilitação que lida com a modelagem e o projeto de sistemas de automação. Originalmente, ela exigia um conhecimento aprofundado de Eletrônica para criar circuitos que implementam esses sistemas. Atualmente, utiliza-se muito da programação de circuitos programáveis, como PLCs e microcontroladores, para implementar os sistemas projetados.
            
        \ssubsubsection{Computação}
        	A área de Computação, por sua vez, lida com programação de computadores (ou de circuitos programáveis, como PLCs ou microcontroladores) e com a utilização de sistemas digitais e da interação entre eles. Assim, existe a computação de \textit{baixo nível}, que lida, por exemplo, com a implementação de protocolos de rede (como o protocolo IP) e com a criação de sistemas operacionais e de compiladores; e de \textit{alto nível}, que está mais relacionada ao desenvolvimento de \textit{softwares}, aplicativos para celulares, programação de jogos, entre outras aplicações. Boa parte da automação hoje em dia é feita através do uso da Computação. Além disso, ela é fortemente utilizada em aplicações que exigem cálculos matemáticos extensos, como na área de inteligência artificial.
            
        \ssubsubsection{Eletrônica}
        	A área de Eletrônica é a base para todas as outras. Nela, estuda-se teoria e prática de circuitos eletroeletrônicos, modelagem de sistemas de controle e aproximações lineares para simplificação de modelos. Além disso, é interessante ter em mente que também aprendemos como circuitos podem implementar funções matemáticas (por exemplo, um circuito cuja saída é a multiplicação entre suas entradas). 
            
            No caso de Controle, utiliza-se de circuitos eletrônicos para a realização de algumas funções matemáticas necessárias para a automação. Para a Computação e Sistemas Digitais, a Eletrônica serve como base para o desenvolvimento do \textit{hardware} -- transistores e resistores são as engrenagens que fazem um computador funcionar.
            
        \ssubsubsection{Sistemas Digitais}
        	Em Sistemas Digitais, lidamos com o estudo e criação de circuitos lógicos e programáveis. Bons exemplos disso são o computador e a calculadora. Além disso, essa área também lida com a ponte entre a Eletrônica e a Computação -- o microprocessador. 
            
            No nosso curso, as disciplinas de Circuitos Lógicos, Sistemas Digitais e Arquitetura de Computadores são a base para essa área, culminando na disciplina de Sistemas Operacionais, além da matéria eletiva Microcomputadores.
            
        \ssubsubsection{Telecomunicações}
        	A área de Telecomunicações lida com a geração, o processamento e a transmissão de sinais. Alguns exemplos de sinais com os quais lidamos no cotidiano são o sinal de TV, sinal de telefone, áudio, vídeo e internet móvel. Nas disciplinas de Sinais e Sistemas (Sistemas Lineares I e II, Processamento de Sinais) aprendemos uma base matemática aprofundada sobre o que é um sinal, como podemos capturar sinais do mundo real e de que forma eles devem ser processados. Já se perguntou por que, às vezes, temos a impressão de que a roda de um carro está girando para trás quando ele está muito rápido? É aqui que você aprende isso e a relação desse fenômeno com os sinais. 
          
            Além dessas disciplinas, temos Comunicações Analógicas e Comunicações Digitais, que ensinam aplicações dessas teorias para sistemas de comunicação. Nelas, aprendemos como é feita a transmissão e recepção de sinais, as limitações que existem em diversos canais de comunicação (ar, cabo, água, etc.) e como lidar com elas de forma eficiente, além de aprender um pouco sobre compressão/descompressão de dados.
            
	\ssubsection{Matérias do 1\textsuperscript{\underline{o}} Período}
    	\ssubsubsection{Cálculo 1}
        
        	Essa matéria pode ser considerada por muitos como o primeiro contato com assuntos muito novos dentro da faculdade. Em geral, aqui você irá aprender um pouco mais sobre funções e algumas propriedades e análises importantes para serem introduzidos os conceitos de derivada e integral. Várias ferramentas muito importantes serão mostradas, como o Teorema do Valor Final, a Regra da Cadeia e a Substituição Trigonométrica em Integrais, temas esses que serão usados em outros momentos no curso.
            
            Por favor, não se assuste! Estamos apresentando temas que serão vistos ao longo da matéria, mas sabemos sim o quanto o novo pode ser difícil. Por conta disso, vamos listar algumas dicas para que você não se sinta tão perdido.
            
            \begin{itemize}
            	\item Está com dúvida? Pergunte! Se não para o seu professor, para algum amigo que você saiba que esteja sabendo a matéria, para um veterano seu que possa estar no GECOM em algum dia que você está estudando por lá; vá a monitorias de Cálculo, os monitores costumam ser bem solícitos
                \item Faça muitos exercícios, você aprende muito com a prática. A Lista do Jair, que pode ser encontrada em qualquer xerox no CT ou no Drive do GECOM, é muito boa, pois tem muitos exercícios de diferentes níveis e, também, gabarito.
                \item A bibliografia do Instituto de Matemática, o livro do Stewart, é vista por muitos como um excelente meio de estudo, pois apresenta teoria e diversos exercícios. Há controvérsias sobre o livro, pois muitos gostam e muitos não gostam, mas é sempre bom dar uma lida e ver se ele atende às suas necessidades, se consegue clarear algum assunto que não ficou bem explicado.
                \item Quando a prova estiver se aproximando, não se esqueça de procurar provas antigas e fazê-las. Vale deixar claro que, muitas vezes, apenas resolver provas antigas não é suficiente (pode ser também, isso varia de pessoa para pessoa).
                \item Há o Responde Aí, uma plataforma paga utilizada por muitos alunos, que tem resumos dos tópicos da matéria, exercícios e também resolução de provas antigas. É interessante, pois a matéria é condensada e bem direcionada para o modelo das provas unificadas do Instituto de Matemática. Porém, vale ressaltar que é uma plataforma direcionada para ensinar a fazer as provas dos Institutos de Física e de Matemática. Sendo assim, aconselhamos o uso deste com esse fim e não com o objetivo de aprender a matéria do zero.
                \item O Departamento de Eletrônica disponibiliza uma Monitoria de Apoio Pedagógico, onde o aluno pode tirar dúvidas de tópicos de matemática do ensino médio, como, por exemplo, números complexos, matrizes e determinantes, trigonometria, etc.
            \end{itemize}
            
        \ssubsubsection{Computação 1}
        	Computação é uma matéria diferente das que são ensinadas no ensino médio, portanto, é necessário que cada um analise, ao longo do período, o quão difícil está sendo a disciplina. Se você sentir que não está entendendo o conteúdo, é importante tentar buscar métodos complementares de estudo (segue lista detalhada abaixo), de modo a facilitar o entendimento durante as aulas. 
            
            Além disso, é muito importante tentar concluir todas as aulas práticas (por mais que você tenha que terminá-las em casa), já que terminá-las significa que você está em dia com a matéria e, para computação, não há jeito melhor de aprender do que fazendo na prática, analisando seus erros e buscando soluções por conta própria.
            
            Lista de fontes alternativas de estudo:
            
            \begin{itemize}
            	\item Coursera: \href{http://www.coursera.com}{http://www.coursera.com}
                \item Udacity: \href{http://www.udacity.com}{http://www.udacity.com}
                \item Udemy: \href{http://www.udemy.com}{http://www.udemy.com}
                \item Learn Python The Hard Way: \href{https://learnpythonthehardway.org/python3/}{https://learnpythonthehardway.org/python3/}
                \item Code Academy: \href{http://codeacademy.com}{http://codeacademy.com}
                \item Canal do \textit{YouTube} de computação da Univesp: \href{https://bit.ly/2uqkS9F}{https://bit.ly/2uqkS9F}
                \item Tutorial no site do \textit{Python}: \href{https://docs.python.org/3/tutorial/}{https://docs.python.org/3/tutorial/}
            \end{itemize}

        \ssubsubsection{Física 1}
        
        Física 1 é considerada uma das matérias mais temidas do primeiro período, mas não deixe isso passar pela sua cabeça, já que, com as dicas que vamos passar aqui, você já vai estar um passo na frente!
        Em Física 1, aprendemos conceitos básicos de mecânica, assim como no ensino médio, mas agora com um grau de dificuldade mais elevado, como vocês já esperam. Vale dizer, também, que os conceitos básicos que vocês irão aprender em cálculo ajudarão na compreensão da matéria. Pode até acontecer de o seu professor de Física 1 precisar ensinar conceitos de cálculo de uma forma mais simplificada, mas não se assuste caso isso ocorra, pois tudo será oficialmente ensinado pelo seu professor de Cálculo 1.
        Uma das diferenças da Física da UFRJ para a do ensino médio é que agora vocês irão passar a manipular expressões algébricas com mais frequência, enquanto, antes, o costume era substituir valores para achar algum resultado.
        
			\begin{itemize}
        		\item É importante encontrar o método de estudo de sua preferência. Geralmente isso não acontece de um dia para o outro, por isso, é importante experimentar os diferentes métodos. Dentre eles, os que se destacam são: Livro \textit{Moysés}, Livro \textit{Young and Freedman}, Livro \textit{Halliday}, Responde Aí e caderno. Existe uma distinção clara entre esses livros, nenhum deles é \textit{o melhor} ou \textit{o pior}, tudo depende do ponto de vista.
                \begin{itemize}
                	\item Moysés: Mais teórico e algébrico para explicar o conteúdo
                	\item Halliday: Costuma usar menos da matemática para explicar os conceitos
                    \item Young and Freedman: Funciona como média entre os outros dois livros, é bem algébrico, mas também dialoga bastante com o leitor.
                	\item Responde Aí: plataforma \textit{online}, descrita anteriormente, com explicações resumidas do conteúdo de Física 1 (entre outras matérias) e provas antigas gabaritadas com resolução. Se decidir usar o Responde Aí, recomendamos utilizar outra ferramenta complementar para estudo (livro, caderno, etc.)
                \end{itemize}
        		\item Resolva provas antigas para estudar, pois, a partir delas, é possível entender o tipo de questão cobrada, assim como se familiarizar com o nível exigido pela UFRJ. Para ter acesso às provas antigas, é possível utilizar nosso \textit{Drive} (ver seção do Drive no manual) ou consultar o site do Instituto de Física, na página da disciplina (só pesquisar ``Física 1 UFRJ'' no \textit{Google}).
                \item É importante tentar se acostumar com a manipulação algébrica, com o funcionamento das unidades de medida (entender o porque da unidade encontrada para cada medida) e com os conceitos de proporcionalidade (inversamente proporcional, diretamente proporcional, etc), já que é costume analisar relações algébricas e entender o que acontece com certas variáveis a partir da mudança de outras.
            	 \item Por fim, mas igualmente importante, é \textbf{ESSENCIAL} que vocês tirem qualquer tipo de dúvida que surgir. Eu sei que parece um discurso de professor de ensino médio, mas, de verdade, não se importe se a sua dúvida parecer muito simples ou banal, é um dever do professor esclarecer qualquer dúvida relacionada ao conteúdo ensinado. Cobre isso dele. 
        	\end{itemize}  
            
        \ssubsubsection{Física Experimental 1}  
        
        O nome dessa matéria é um pouco intuitivo, mas a ideia é pôr em prática alguns dos conceitos vistos nas aulas de Física I. São feitos experimentos dentro dos laboratórios do Bloco A, no 4\textsuperscript{\underline{o}} andar (são alguns bons lances de escada até lá, então evite chegar em cima da hora), dentre os quais são estudados conceitos de colisões elásticas e inelásticas, aceleração da gravidade (sim, você mede a gravidade!), movimento de corpo rígido, entre outros.
        
        Há alguns cuidados a serem tomados, pois professores de Fisexp geralmente são bem rígidos com algumas regras. Dessa forma, vamos listar algumas dicas:
        
        \begin{itemize}
        	\item Não chegue atrasado. É possível reprovar por falta e, considerando que há a possibilidade de você ter aula às 8h da manhã, programe-se bem, pois eles levam muito a sério a tolerância de atraso.
            \item Seu professor irá ensiná-lo a tirar medidas da forma correta, com suas respectivas incertezas. Preste muita atenção nisso e, em caso de dúvida, pergunte. A incerteza de uma medida é muito importante e é usada em todas as Físicas Experimentais que temos. Por mais bobo que pareça, não negligencie esse assunto, ok?
        \end{itemize}
        \ssubsubsection{História da Tecnologia}
        	É a matéria mais tranquila do primeiro período, a ementa gira em torno de textos e filmes que o professor passa, e pede para que você faça uma "reação", que nada mais é que dois parágrafos comentando sua visão daquele texto/filme. Além disso, o professor chama representantes de vários laboratórios do departamento e de equipes de competição do CT para dar mini-palestras. A presença é obrigatória e a nota é dada baseada na presença e nas reações.
            
        \ssubsubsection{Química}
        	A ementa e o nível que ela é cobrada, em geral, dependem muito do professor. Existem alguns materiais no \textit{drive}, mas a maior dica é realmente acompanhar a aula do seu professor, não costuma ser uma matéria problemática. A avaliação costuma ser duas provas, P1 e P2, passando direto se a média das provas for maior ou igual a 7, caso contrário, você faz uma prova final (PF) e está aprovado se a média entre PF e P1+P2 for, no mínimo, 5.
     
    
    \ssubsection{Laboratórios}
		Nosso curso apresenta vários laboratórios, onde professores desenvolvem projetos e também orientam alunos de Iniciação Científica (falaremos adiante), mestrado e doutorado. São muito importantes para o curso, pois há desenvolvimento de novas tecnologias e podemos estar em contato com isso. Para cada um dos principais, será feita uma breve descrição das atividades.

        \ssubsubsection{Grupo de Teleinformática e Automação (GTA)}
        	O Grupo de Teleinformática e Automação (GTA) iniciou suas atividades em março de 1986. A atuação do GTA tem-se dado tanto na graduação, no Departamento de Engenharia Eletrônica e de Computação (DEL) da Escola Politécnica (Poli) da UFRJ, como na pós-graduação, no Programa de Engenharia Elétrica (PEE) da Coordenação de Programas de Pós-Graduação em Engenharia (COPPE) da UFRJ. O GTA tem realizado diversos trabalhos em conjunto com equipes de redes de computadores do Brasil e do exterior.

        	O GTA fornece treinamento e presta serviços na área de sistemas de comunicação. O treinamento está voltado para profissionais das áreas de informática e telecomunicações, como gerentes de projetos, projetistas, analistas, engenheiros e usuários finais. O objetivo é apresentar a estes profissionais as mais novas tecnologias existentes. Além disto, utiliza sua vasta experiência na área para prestar serviços de consultoria, projeto, implantação, gerência e manutenção de sistemas de teleinformática.

        \href{https://www.gta.ufrj.br/}{Site: https://www.gta.ufrj.br/}
        
        \ssubsubsection{Laboratório de Modelos de Computação e Inteligência de Máquina (IM$^2$C)} 
        	O Laboratório de Modelos de Computação e Inteligência de Máquina
        aborda temas relacionados com computação pura e aplicada, como:
        inteligência artificial, mineração de dados, cibersegurança,
        criptografia, teoria da computabilidade, sistemas especialistas,
        sistemas de apoio à decisão, predição de séries temporais, modelos
        computacionais de processamento e aprendizado de máquina.
        
        \ssubsubsection{Laboratório de Processamento Analógico e Digital de Sinais (PADS)}
        	O Laboratório de Processamento Analógico e Digital de Sinais (PADS) foi criado em 1997 e, desde então, vem atuando em atividades de pesquisa e de ensino nos níveis de graduação, em conjunto com o Departamento de Engenharia Eletrônica e de Computação (DEL) da Escola Politécnica, e de pós-graduação, em conjunto com o Programa de Engenharia Elétrica (PEE) da COPPE.

        	Entre as principais áreas de pesquisa dos docentes, alunos e pesquisadores do PADS estão o Processamento Analógico e Digital de Sinais e a Microeletrônica Analógica e de Radiofrequência. Para atender a essas duas áreas de pesquisa, o PADS conta com dois espaços localizados no Bloco H do Centro de Tecnologia: a sala H-320, onde são realizadas as pesquisas na área de Processamento de Sinais, e a sala H-206, onde são realizadas as pesquisas na área de Microeletrônica.

        	Os docentes do curso de Engenharia Eletrônica e de Computação que atuam como pesquisadores do PADS são:   
             \begin{itemize}
              \item[-] Antonio Petraglia (PhD)
              \item[-] Carlos Fernando Teodósio Soares (D.Sc) 
              \item[-] Fernando Antonio Pinto Barúqui (D.Sc)
              \item[-] Joarez Bastos Monteiro (D.Sc)
              \item[-] José Gabriel Rodriguez Carneiro Gomes (PhD)
              \item[-] Mariane Rembold Petraglia (PhD)
             \end{itemize}     
        	Entre as principais linhas de pesquisa na área de Processamento de Sinais, podemos citar: Filtragem Adaptativa de Sinais e Processamento de Imagens. Já as principais linhas de pesquisa na área de Microeletrônica são: Filtros Analógicos Contínuos e Discretos no Tempo Integrados em Tecnologia CMOS, Circuitos Coletores de Energia, Circuitos Conversores Analógico/Digital e Imageadores CMOS.

        \href{http://www.pads.ufrj.br/}{Site: http://www.pads.ufrj.br/}

        \ssubsubsection{Laboratório de Processamento de Sinais (LPS)}
        	As principais áreas de atuação do LPS são: Instrumentação Eletrônica; Microeletrônica Analógica, Filtros Elétricos e Circuitos de Radiofrequência, Processamento de Sinais de Voz, Processamento Estocástico, Redes Neurais, Sistemas Web de Gerência da Informação, Sistemas Web de Processamento.

        \href{http://www.lps.ufrj.br/}{Site: http://www.lps.ufrj.br/}

        \ssubsubsection{Laboratório de Sinais, Multimídia e Telecomunicações (SMT)}
        	Criado em 1996 na Universidade Federal do Rio de Janeiro (UFRJ), o Laboratório de Sinais, Multimídia e Telecomunicações (SMT), anteriormente parte do Laboratório de Processamento de Sinais (LPS), reúne um grupo de pesquisadores internacionalmente reconhecidos que trabalha ativamente em pesquisa acadêmica e educação, tanto no nível de Pós-Graduação, no Instituto Alberto Luiz Coimbra de Pós-Graduação e Pesquisa de Engenharia (COPPE/UFRJ), quanto no nível de Graduação, na Escola Politécnica (Poli/UFRJ), bem como em consultoria.

        	O SMT, atualmente, congrega 7 professores (sendo 4 titulares) em tempo integral, além de pesquisadores de pós-doutorado, alunos de pós-graduação e alunos de graduação. As atividades do SMT abrangem diversos objetivos, com foco em ensino e pesquisa de Engenharia Elétrica, serviços de consultoria para a indústria e disseminação de conhecimento em ferramentas modernas de processamento de sinais, tanto no Brasil como no exterior.

        	Entre as principais linhas de pesquisa, é possível destacar: Processamento de sinais aleatórios, de sinais de áudio, processamento de imagens, processamento de fala, processamento adaptativo de sinais, inteligência computacional, compressão de sinais, entre outros.

        \href{https://www.smt.ufrj.br/pt/}{Site: https://www.smt.ufrj.br/pt/}