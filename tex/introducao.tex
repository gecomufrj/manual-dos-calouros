\ssection{Introdução}
	O GECOM apresenta o Manual do Calouro do DEL. Aqui, tentamos condensar algumas das principais informações necessárias para você não passar apuros no seu primeiro período na UFRJ. Se você está lendo isso, provavelmente não tem ideia de que GTA pode não significar \textit{Grand Theft Auto}, não sabe o quão importante é ter um rolo de papel higiênico na mochila e acha que 15 reais é barato para um prato de comida. 
    Claramente, na prática a teoria é outra, mas pelo menos a gente tentou.
    
   	Incluindo até mesmo lições de LaTex durante noites vagas para elaborar esse manual, deixamos aqui nossa satisfação em apresentá-los para vocês. Assim, desejamos uma excelente leitura e esperamos que você aproveite. 
    
    Aproveitamos esse momento para agradecer aos nossos colaboradores, que fizeram um excelente trabalho de pesquisa e elaboração de um compilado de linhas de ônibus, para ajudá-los nessa nossa jornada rumo ao país Fundão. São eles: Gabriel de Lima, Karen Pacheco, Mario Simão e Gabriel Parracho.
    
    Gostaríamos também de deixar nossos agradecimentos ao nosso coordenador de curso, Carlos Teodósio, que nos ajudou revisando o material e nos explicando um pouco mais sobre os processos burocráticos do curso.
    
    Por fim, gostaríamos de agradecer pelo \textit{feedback} positivo e ideias de melhorias de diversos alunos que leram a primeira versão desse manual, lançada em 2018.2. São eles: Bernardo Boechat, Felipe Vianna, Gabriela Dantas e Nathalia Chrispim.
    
    Um excelente período para todos,
    
    \begin{itemize}
    	\item [] Fabiana Ferreira    
    	\item [] Gabriel Romão
		\item [] Lucas Cerqueira
		\item [] Luiz Giserman
		\item [] Paulo Valente
    \end{itemize}
    